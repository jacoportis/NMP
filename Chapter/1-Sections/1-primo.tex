Supponiamo di avere $n+1$ dati $(x_i,y_i)$ con $i=0,\ldots,n$

$$\chi^2(\underline{a})
=\frac{1}{n-1} \sum_{i=0}^{n} \big[ y_i - f(\underline{x}_i,\underline{a}) \big]^2\geq0$$

$$\begin{cases}
   c_0 + c_1x_0 + c_2 x_0^2 + \ldots + c_n x_0^n=y_0\\
   c_0 + c_1x_1 + c_2 x_1^2 + \ldots + c_n x_1^n=y_1\\
   \vdots\\
   c_0 + c_1x_n + c_2 x_n^2 + \ldots + c_n x_n^n=y_n
\end{cases}$$

Possiamo esprimere il sistema come

$$V \underline{c}=\underline{y}$$

dove

$$V=
\begin{pmatrix}
   1 & x_0 & x_0^2 & \ldots & x_0^n\\
   1 & x_1 & x_1^2 & \ldots & x_1^n\\
   \vdots & \vdots & \vdots & \ddots & \vdots\\
   1 & x_n & x_n^2 & \ldots & x_n^n\\
\end{pmatrix}
\quad,\quad
\underline{c}=
\begin{pmatrix}
   c_0\\
   \vdots\\
   c_n
\end{pmatrix}
\quad,\quad
\underline{y}=
\begin{pmatrix}
   y_0\\
   \vdots\\
   y_n
\end{pmatrix}$$

$V$ è detta \textbf{matrice di Vandermonde}. Poiché $x_i \neq x_j \quad \forall \, i \neq j$, ogni colonna è diversa e quindi il determinante è non nullo. Per calcolarlo sottraiamo ad ogni colonna la precedente moltiplicata per $x_0$:

$$\det (V)=
\begin{pmatrix}
   1 & x_0 - x_0 & x_0^2 - x_0^2 & \ldots & x_0^n - x_0^n\\
   1 & x_1 - x_0 & x_1^2 - x_0x_1 & \ldots & x_1^n - x_0x_1^{n-1}\\
   \vdots & \vdots & \vdots & \ddots & \vdots\\
   1 & x_n - x_0 & x_n^2 - x_0x_n & \ldots & x_n^n - x_0 x_n^{n-1}\\
\end{pmatrix}$$

$$\implies \det (V)=
\begin{pmatrix}
   1 & 0 & 0 & \ldots & 0\\
   1 & x_1 - x_0 & (x_1 - x_0)x_1 & \ldots & (x_1 - x_0)x_1^{n-1}\\
   \vdots & \vdots & \vdots & \ddots & \vdots\\
   1 & x_n - x_0 & (x_n - x_0)x_n & \ldots & (x_n - x_0)x_n^{n-1}\\
\end{pmatrix}$$

dividendo la $j$-esima riga (tranne la prima) per $x_j - x_0$ e portando fuori dalla matrice otteniamo

$$\det (V)=
\begin{pmatrix}
   1 & 0 & 0 & \ldots & 0\\
   1 & 1 & x_1 & \ldots & x_1^{n-1}\\
   \vdots & \vdots & \vdots & \ddots & \vdots\\
   1 & 1 & x_n & \ldots & x_n^{n-1}\\
\end{pmatrix}
\underbrace{(x_1 - x_0) \cdot \ldots \cdot (x_n - x_0)}_{\displaystyle = \prod_{j=2}^{n} (x_j - x_0)}$$

La matrice risultante non dipende più da $x_0$. Reiterando il processo possiamo eliminare la dipendenza da ogni $x_i$ e in forma compatta il determinante si potrà scrivere come

$$\det V=\prod_{i>j} (x_i - x_j)$$

Tale espressione ci dice che se i punti sono tutti distinti il determinante della matrice è diverso da zero. Si può anche scrivere come

$$\det V=\prod_{j=0}^{n-1} \qty( \prod_{i=j+1}^{n} (x_i - x_j) )$$

\textbf{non ho capito come si passa a questo vabbè}

$$P_n(x)=\sum_{i=0}^{n} f(x_i) L_i(x)$$

dove

$$L_i(x)=\prod_{\begin{subarray}{c}
   k=0\\[0.1cm]
   k \neq i
\end{subarray}
}^{n} \frac{x - x_k}{x_i - x_k}
$$

nota: essendo così definito, il denominatore è sempre diverso da zero.

L'ordine di ogni polinomio $L_i$ è proprio $n$

$$L_i=\frac{
(x - x_0) (x - x_1) \ldots (x - x_{i-1}) (x - x_{i+1}) \ldots (x - x_n)
}{
(x_i - x_0) (x_i - x_1) \ldots (x_i - x_{i-1}) (x_i - x_{i+1}) \ldots (x_i - x_n)
}$$

Si ha che

$$L_i(x_j)=
\left\{
   \begin{array}{ll}
      0 & \text{se } j \neq i\\
      1 & \text{se } j = i
   \end{array}
\right.
\equiv \delta_{ij}$$

infatti nel primo caso a numeratore ci sarà un termine nullo, nel secondo caso numeratore e denominatore saranno uguali.

In definitiva abbiamo che

$$P_n(x_j)=\sum_{i=0}^{n} f(x_i) L_i(x_j)
=\sum_{i=0}^{n} f(x_i) \delta_{ij}
=f(x_j) \quad \forall \, j$$

\subsection{Formula di Newton per il polinomio interpolante}

Tale metodo si adopera quando abbiamo trovato la soluzione per $n$ punti e ne includiamo un altro. Supposto quindi di avere il polinomio $P_{n-1}(x)$ che passa per $x_0,x_1,\ldots,x_{n-1}$, poniamo

$$P_n(x)=P_{n-1}(x) + a_n w_{n-1}(x)$$

dove

$$w_{n-1}(x)=\prod_{i=0}^{n-1} (x - x_i)
\quad,\quad
a_n=\frac{f(x_n) - P_{n-1}(x_n)}{w_{n-1}(x_n)}$$

$w_{n-1}$ è un \textit{polinomio monico}, cioè un polinomio tale che il coefficiente del termine di grado massimo è pari a 1. Inoltre ha grado $n$.

\vspace{0.2cm}Proviamo innanzitutto che il nuovo polinomio passa per i primi $n-1$ punti.

Posto $k \leq n-1$ si ha

$$w_{n-1}(x_k)=0$$
$$P_n(x_k)=P_{n-1}(x_k) + a_n \underbrace{w_{n-1}(x_k)}_{=0}
=P_{n-1}(x_k)
\equiv f(x_k)$$

Proviamo adesso che passa per il nuovo punto:

$$P_n(x_n)=P_{n-1}(x_n) + a_n w_{n-1}(x_n)
=P_{n-1}(x_n) + \frac{f(x_n) - P_{n-1}(x_n)}{w_{n-1}(x_n)} w_{n-1}(x_n)=$$
$$=P_{n-1}(x_n) + f(x_n) - P_{n-1}(x_n)
=f(x_n) $$